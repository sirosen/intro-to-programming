%!TEX root = main.tex

\titlebox{Part 1: Getting Set-Up}

\subtitlebox{Find the Terminal}
Go to Spotlight Search or Applications and find the Terminal app.
Many people prefer the freely available iTerm app, but both will behave the same for our purposes.
Once you have launched the terminal, you will be presented with a command prompt, which is a simple text entry field.
You can now type commands into the terminal, such as \texttt{ls} to list the contents of the current working directory.
Do not worry too much about using the terminal -- we will revisit the commands available there later on.

\subtitlebox{Make Sure Python is Installed}
Type \texttt{python} into the command prompt (the terminal's text field), and hit Return.
You should be presented with something like this:
\begin{codeblock}
Python 2.7.3 (default, Aug  1 2012, 05:14:39)
[GCC 4.6.3] on linux2
Type "help", "copyright", "credits" or "license" for more information.
>>>
\end{codeblock}
Don't worry too much about the specifics.
The important part is that this program, the Python interpreter, launches.
You will notice that the text field into which you typed \texttt{python} was preceded by text ending in a \texttt{\$} symbol (we will discuss the meaning of this text when we cover Terminal commands), while the new field is not.
Instead, you have the new text field or prompt, \pyprompt.
We will call this the ``python prompt'' or the ``interpreter'', but these both refer to the \pyprompt text field.
If python does not launch, your system is probably in some kind of trouble.
You can install it from \href{http://www.python.org/getit/mac/}{Python.org}.

\subtitlebox{Exit the Interpreter, and Quit Terminal}
Let's make sure we're comfortable with launching python.
\begin{itemize}
    \item Hit ``Ctrl-D'' or enter \texttt{exit()} to exit the python interpreter.
    \begin{itemize}
        \item Ctrl-D is the ``EOF'' or ``End of File'' character, a special character which indicates the end of a file.
            You can conceive of characters as keystrokes that may or may not be possible to enter on a standard keyboard, and Ctrl-D as being equivalent to typing a special key, distinct from D.
        \begin{itemize}
            \item Python interprets the EOF as a signal to exit.
        \end{itemize}
    \item \texttt{exit()} is a ``function call''. Essentially, it is the name of some piece of code which we would like to run.
        \begin{itemize}
            \item Python evaluates \texttt{exit()}, which involves running the function named \texttt{exit}, which exits the interpreter
            \item For now, you can think of \texttt{exit()} as being a magical command that exits the interpreter. We will talk in depth about functions soon
        \end{itemize}
    \end{itemize}
    \item Quit Terminal just like any application. Command+Q, or the menu
    \begin{itemize}
        \item If we had not closed the interpreter before quitting, Terminal may have complained that a program was still being run by the Terminal
    \end{itemize}
\end{itemize}


