Type \texttt{python} into the command prompt (the terminal's text field), and hit Return.
You should be presented with something like this:
\begin{codeblock}
Python 2.7.3 (default, Aug  1 2012, 05:14:39)
[GCC 4.6.3] on linux2
Type "help", "copyright", "credits" or "license" for more information.
>>>
\end{codeblock}
Don't worry too much about the specifics.
The important part is that this program, the Python interpreter, launches.
You will notice that the text field into which you typed \texttt{python} was preceded by text ending in a \texttt{\$} symbol (we will discuss the meaning of this text when we cover Terminal commands), while the new field is not.
Instead, you have the new text field or prompt, \pyprompt.
We will call this the ``python prompt'' or the ``interpreter'', but these both refer to the \pyprompt text field.
If python does not launch, your system is probably in some kind of trouble.
You can install it from \href{http://www.python.org/getit/mac/}{Python.org}.
