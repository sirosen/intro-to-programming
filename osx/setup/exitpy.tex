Let's make sure we're comfortable with launching python.
\begin{itemize}
    \item Hit ``Ctrl-D'' or enter \texttt{exit()} to exit the python interpreter.
    \begin{itemize}
        \item Ctrl-D is the ``EOF'' or ``End of File'' character, a special character which indicates the end of a file.
            You can conceive of characters as keystrokes that may or may not be possible to enter on a standard keyboard, and Ctrl-D as being equivalent to typing a special key, distinct from D.
        \begin{itemize}
            \item Python interprets the EOF as a signal to exit.
        \end{itemize}
    \item \texttt{exit()} is a ``function call''. Essentially, it is the name of some piece of code which we would like to run.
        \begin{itemize}
            \item Python evaluates \texttt{exit()}, which involves running the function named \texttt{exit}, which exits the interpreter
            \item For now, you can think of \texttt{exit()} as being a magical command that exits the interpreter. We will talk in depth about functions soon
        \end{itemize}
    \end{itemize}
    \item Quit Terminal just like any application. Command+Q, or the menu
    \begin{itemize}
        \item If we had not closed the interpreter before quitting, Terminal may have complained that a program was still being run by the Terminal
    \end{itemize}
\end{itemize}
